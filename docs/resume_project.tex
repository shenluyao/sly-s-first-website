\textbf{借助 Vibe Coding 搭建个人网站} \hfill 2026.01
\titlebreak
项目描述:为了管理个人学习笔记,记录每日学习任务,并对访问人数进行统计,基于 \textbf{TypeScript} 与 \textbf{Next.js} 框架搭建了一个全栈个人网站,部署于 \textbf{Vercel} 平台。
\begin{itemize}[nosep, topsep=\intraProjectListTopSep]
    \item \textbf{待办清单(To-Do List)}:运用 \textbf{TypeScript} 语言及 \textbf{Next.js} App Router 框架,结合 \textbf{Supabase}(PostgreSQL)数据库,实现了每日待办事项的\textbf{增、删、改、查}功能,支持任务完成状态切换与完成进度百分比的\textbf{实时统计},并采用\textbf{乐观更新}策略提升用户交互体验。
    \item \textbf{笔记管理}:基于 \textbf{Supabase Storage} 实现了学习笔记的\textbf{文件上传}(支持 PDF 与 Markdown 格式)、\textbf{在线预览}与\textbf{删除}功能;利用 \textbf{React Markdown} 在模态框内渲染 Markdown 笔记,PDF 文件支持新标签页打开阅读。
    \item \textbf{访问统计}:基于 \textbf{Redis}(Vercel KV)实现了页面\textbf{访问计数器},通过 \textbf{Cookie} 机制进行访客去重(7 天窗口期),并利用 \texttt{useRef} 防止 React Strict Mode 下的重复计数,确保统计数据的准确性。
    \item \textbf{技术栈}:前端采用 \textbf{Tailwind CSS} 构建\textbf{响应式}界面,后端使用 Next.js \textbf{Server Actions} 处理业务逻辑,数据层使用 \textbf{Supabase} 与 \textbf{Redis},整体部署于 \textbf{Vercel} 云平台。
\end{itemize}
